\documentclass[english,10pt,aspectratio=169,fleqn]{beamer}

\usepackage{amsmath} % load this before unicode-math
\usepackage{amssymb}
\usepackage{mathabx}
%\usepackage{unicode-math}

\usepackage{fontspec}
\setmonofont{DejaVu Sans Mono}
%\setmathfont{STIXMath}
%\setmathfont{TeX Gyre Termes Math}

\usefonttheme[onlymath]{serif}

\setlength{\parskip}{\smallskipamount}
\setlength{\parindent}{0pt}

%\setbeamersize{text margin left=5pt, text margin right=5pt}

\usepackage{amsmath}
\usepackage{amssymb}
\usepackage{braket}

\usepackage{minted}
\newminted{julia}{breaklines,fontsize=\scriptsize,texcomments=true}
\newminted{python}{breaklines,fontsize=\scriptsize,texcomments=true}
\newminted{bash}{breaklines,fontsize=\scriptsize,texcomments=true}
\newminted{text}{breaklines,fontsize=\scriptsize,texcomments=true}

\newcommand{\txtinline}[1]{\mintinline[fontsize=\scriptsize]{text}{#1}}
\newcommand{\jlinline}[1]{\mintinline[fontsize=\scriptsize]{julia}{#1}}

\definecolor{mintedbg}{rgb}{0.95,0.95,0.95}
\usepackage{mdframed}

%\BeforeBeginEnvironment{minted}{\begin{mdframed}[backgroundcolor=mintedbg]}
%\AfterEndEnvironment{minted}{\end{mdframed}}

\setcounter{secnumdepth}{3}
\setcounter{tocdepth}{3}

\makeatletter

 \newcommand\makebeamertitle{\frame{\maketitle}}%
 % (ERT) argument for the TOC
 \AtBeginDocument{%
   \let\origtableofcontents=\tableofcontents
   \def\tableofcontents{\@ifnextchar[{\origtableofcontents}{\gobbletableofcontents}}
   \def\gobbletableofcontents#1{\origtableofcontents}
 }

\makeatother

\usepackage{babel}
\usepackage{braket}

\begin{document}


\title{PAW Notes}
\author{ffr}
\institute{}
\date{}

\frame{\titlepage}


\begin{frame}

\begin{equation*}
V_{\mathrm{NL}} = \sum_{nm,I} D^{(0)}_{nm} \ket{\beta^{I}_{n}}\bra{\beta^{I}_{m}}
\end{equation*}

$n,m$: beta func indices

$\beta^{I}_{n}(\mathbf{r}) = \beta_{n}(\mathbf{r} - \mathbf{R}_{I})$

Electron density:
\begin{equation*}
n(\mathbf{r}) = \sum_{i} \left[
| \psi_{i}(\mathbf{r}) |^{2} +
\sum_{nm,I} Q^{I}_{nm}(\mathbf{r}) \braket{ \psi_{i} | \beta_{n} }
\braket{ \beta_{m} | \psi_{i} }
\right]
\end{equation*}

\end{frame}


\begin{frame}

Orthonormality conditions:
\begin{equation*}
\braket{\psi_{i} | S | \psi_{j} } = \delta_{ij}
\end{equation*}
overlap operator:
\begin{equation*}
S = 1 + \sum_{nm,I} q_{nm} \ket{\beta^{I}_{n}} \bra{\beta^{I}_{m}}
\end{equation*}

\begin{equation*}
q_{nm} = \int Q_{nm}(\mathbf{r})\,\mathrm{d}\mathbf{r}
\end{equation*}

\end{frame}



\begin{frame} % --------------------------

KS equation:
\begin{equation*}
H\ket{\psi_{i}} = \epsilon_{i} S \ket{\psi_{i}}
\end{equation*}

Hamiltonian operator:
\begin{equation*}
H = -\frac{\hbar^2}{2m} \nabla^2 + V_{\mathrm{eff}} +
\sum_{nm,I} D^{I}_{nm} \ket{\beta^{I}_{n}} \bra{\beta^{I}_{m}}
\end{equation*}

Screened coefficients:
\begin{equation*}
D^{I}_{nm} = D^{(0)}_{nm} + \int V_{\mathrm{eff}}(\mathbf{r})
Q^{I}_{nm}(\mathbf{r})\,\mathrm{d}\mathbf{r}
\end{equation*}

\end{frame}


\begin{frame}

\begin{equation*}
E = \sum_{n} f_{n} \Braket{\psi_{n} | -\frac{1}{2}\nabla^2 | \psi_{n}} +
E_{\mathrm{H}}\left[ n + n_{Z} \right] + E_{\mathrm{xc}}\left[ n \right]
\end{equation*}

\begin{equation*}
\ket{\psi_{n}} = \ket{\tilde{\psi}_{n}} + \sum_{i} \left(
\ket{\phi_{i}} - \ket{\tilde{\phi}_{i}} \right)
\braket{ \tilde{p}_{i} | \tilde{\psi}_{n} }
\end{equation*}

Index $i$: atomic site $\mathbf{R}$, angular momentum  $L = (l,m)$, and additional
index $k$ for reference energy $\epsilon_{kl}$.

AE partial waves $\phi_{i}$ are obtained for a reference atom

PS partial waves $\tilde{\phi}_{i}$ are equivalent to the AE partial waves outside
a core radius $r^{l}_{c}$ and match continuously onto $\tilde{\phi}_{i}$ inside the
core radius.

Projector functions $\tilde{p}_{i}$ are dual to the partial waves:
\begin{equation*}
\braket{\tilde{p}_{i} | \tilde{\phi}_{j}} = \delta_{ij}
\end{equation*}

\end{frame}


\begin{frame} % ----------------------

\begin{equation*}
n(\mathbf{r}) = \tilde{n}(\mathbf{r}) + n^{1}(\mathbf{r}) - \tilde{n}^{1}(\mathbf{r})
\end{equation*}

\begin{equation*}
\tilde{n}(\mathbf{r}) = \sum_{n} f_{n} \braket{\tilde{\psi}_{n} | \mathbf{r}}
\braket{\mathbf{r} | \tilde{\psi}_{n}}
\end{equation*}

Onsite electron densities $n^{1}$ and $\tilde{n}^{1}$ are treated on a radial
support grid, that extends up to $r_{\mathrm{rad}}$ around each ion.

\begin{equation*}
n^{1}(\mathbf{r}) = \sum_{i,j} \rho_{ij} \braket{\phi_{i} | \mathbf{r}}
\braket{\mathbf{r} | \phi_{j}}
\end{equation*}

\begin{equation*}
\tilde{n}^{1}(\mathbf{r}) = \sum_{i,j} \rho_{ij} \braket{\tilde{\phi}_{i} | \mathbf{r}}
  \braket{\mathbf{r} | \tilde{\phi}_{j}}
\end{equation*}


\end{frame}


\begin{frame} % -----------------------------

$\rho_{ij}$ are the  occupancies of each augmentation channel $(i,j)$:
\begin{equation*}
\rho_{ij} = \sum_{n} f_{n} \braket{\tilde{\psi}_{n} | \tilde{p}_{i} }
\braket{\tilde{p}_{i} | \tilde{\psi}_{n} }
\end{equation*}

\end{frame}



\begin{frame}

Assuming frozen core approximation.

Introduce four quantities that will be used for the description of the core
charge density:

$n_{c}$: charge density of frozen core all-electron wave functions
in the reference atom.

$\tilde{n}_{c}$: partial electronic core density $\tilde{n}_{c}$ is equivalent
to the frozen core AE charge density outside a certain radius $r_{pc}$, which lies
inside the argumentation radius. Used in order to calculate nonlinear core
corrections

$n_{Zc}$: point charge density of the nuclei $n_{Z}$ plus the frozen core AE charge
density $n_{c}$:
\begin{equation*}
n_{Zc} = n_{Z} + n_{c}
\end{equation*}

$\tilde{n}_{Zc}$: pseudized core density, equivalent to $n_{Zc}$ outside the core radius
and shall have the same moment as $n_{Zc}$ inside the core region:
\begin{equation*}
\int_{\Omega_{r}} n_{Zc}(\mathbf{r}) \, \mathrm{d}\mathbf{r} =
\int_{\Omega_{r}} \tilde{n}_{Zc}(\mathbf{r}) \, \mathrm{d}\mathbf{r}
\end{equation*}

\end{frame}


\end{document}

